\documentclass[runningheads]{llncs}
\usepackage[T1]{fontenc}
\usepackage{graphicx}
\usepackage{hyperref}
\usepackage{color}
\usepackage{amsmath,amsfonts}
\usepackage{algorithm}
\usepackage{algpseudocode}
\usepackage{caption}
\usepackage{multirow}
\usepackage{tikz}
\usetikzlibrary{positioning}

% Custom commands
\newcommand{\pke}{\texttt{PKE}}
\newcommand{\keygen}{\texttt{Gen}}
\newcommand{\encrypt}{\texttt{Enc}}
\newcommand{\decrypt}{\texttt{Dec}}
\newcommand{\kem}{\texttt{KEM}}
\newcommand{\encap}{\texttt{Encap}}
\newcommand{\decap}{\texttt{Decap}}
\newcommand{\etm}{\texttt{EtM}}  % encrypt-then-mac
\newcommand{\mac}{\texttt{MAC}}
\newcommand{\sign}{\texttt{Sign}}
\newcommand{\verify}{\texttt{Verify}}
\newcommand{\pk}{\texttt{pk}}
\newcommand{\sk}{\texttt{sk}}
\newcommand{\pco}{\texttt{PCO}}
\newcommand{\cvo}{\texttt{CVO}}
\newcommand{\leftsample}{\stackrel{\$}{\leftarrow}}
\newcommand{\llbrack}{[\![}
\newcommand{\rrbrack}{]\!]}
\newcommand{\norm}[1]{\left\lvert #1 \right\rvert}
\newcommand{\adv}{\texttt{Adv}}
\newcommand{\fotplus}{\texttt{FOT+}}
\newcommand{\us}{\mu s}
\newcommand{\wt}{\mathop{wt}}
\def\mlkemplus{\text{ML-KEM}^+}

\begin{document}

\title{Faster generic CCA secure KEM transformation using encrypt-then-MAC}
%\titlerunning{Abbreviated paper title}

\author{
    Ganyu Xu\inst{1} \and
    Guang Gong\inst{1} \and
    Kalikinkar Mandal\inst{2}
}
% First names are abbreviated in the running head.
% If there are more than two authors, 'et al.' is used.
\authorrunning{G. Xu et al.}

\institute{
    University of Waterloo, Waterloo, Ontario, Canada
    \email{\{g66xu,ggong\}@uwaterloo.ca} \and
    University of New Brunswick, Canada \email{kmandal@unb.ca}
}

\maketitle              % typeset the header of the contribution
%
\begin{abstract}
    TODO: write abstract later
    \keywords{
    First keyword  \and Second keyword \and Another keyword.
}
\end{abstract}

\section{Introduction}\label{sec:introduction}
Key encapsulation mechanism (KEM) is a public-key cryptographic primitive that allows two parties to establish a shared secret over an insecure communication channel. The accepted security requirement of a KEM is \textit{Indistinguishability under adaptive chosen ciphertext attack (IND-CCA)}. Intuitively speaking, IND-CCA security implies that no efficient adversary (usually defined as probabilistic polynomial time Turing machine) can distinguish a pseudorandom shared secret from a uniformly random bit string of identical length even with access to a decapsulation oracle. Unfortunately, CCA security is difficult to achieve from scratch. Early attempts at constructing CCA secure public-key cryptosystems using only heuristics argument and without using formal proof, such as RSA encryption in PKCS \#1 \cite{rfc2313} and RSA signature ISO 9796 \cite{ISO9796-1}, were badly broken with sophisticated cryptanalysis \cite{DBLP:conf/crypto/Bleichenbacher98,coppersmith1999iso,DBLP:conf/crypto/CoronNS99}. Afterwards, provable chosen ciphertext security became a necessity for new cryptographic protocols. There have been many provable CCA secure constructions since then. Notable examples include Optimal Asymmetric Encryption Padding (OAEP) \cite{DBLP:conf/eurocrypt/BellareR94}, which is combined with RSA \cite{DBLP:conf/crypto/FujisakiOPS01} into the widely adopted RSA-OAEP. The Fujisaki-Okamoto transformation \cite{DBLP:conf/crypto/FujisakiO99,DBLP:conf/tcc/HofheinzHK17} is another generic CCA secure transformation that was thoroughly studied and widely adopted, particularly by many KEM candidates in NIST's Post Quantum Cryptography (PQC) standardization project.

Chosen ciphertext security is a solved problem within the context of symmetric cryptography. It is well understood that authenticated encryption can be achieved by combining a semantically secure symmetric encryption scheme with an existentially unforgeable message authentication code (MAC) using either the ``encrypt-then-MAC'' (AES-GCM, ChaCha20-Poly1305) or ``MAC-then-encrypt'' pattern (AES-CCM)\cite{DBLP:conf/asiacrypt/BellareN00,DBLP:conf/crypto/Krawczyk01}. However, adapting this technique for public-key cryptosystems is challenging, since the two communicating parties do not have a pre-shared symmetric key. The first attempt at such adaption is the Diffie-Hellman integrated encryption scheme (DHIES) \cite{DBLP:journals/iacr/AbdallaBR99,DBLP:conf/ctrsa/AbdallaBR01} proposed by Abdalla, Bellare, and Rogaway, who proved its chosen ciphertext security under a non-standard but well studied assumption called ``Gap Diffie-Hellman problem'' \cite{DBLP:conf/pkc/OkamotoP01}. DHIES was standardized in IEEE P1363a \cite{P1363a-2004}.

\subsection{Our contribution}

\bibliographystyle{splncs04}
\bibliography{biblio.bib}
\end{document}
